
\documentclass[letterpaper,11pt]{article}

\usepackage{latexsym}
\usepackage[fleqn]{amsmath}
\usepackage{amssymb}
\usepackage{unicode-math}
\usepackage{fancyhdr}
\usepackage[margin=1.0in, left=0.5in, right=0.5in, top=1.25in, headsep=10mm, headheight=15mm]{geometry}
\usepackage{graphicx}

\setmainfont{Latin Modern Roman}
\setmathfont{Latin Modern Math}

\pagestyle{fancy}
\rhead{Mech 222 Mathematics Worksheets\\ Lecture 13, 14}


\newcommand*\limitset[1]{{#1}^\prime}
% Not working in unicode math with xelatex for some reason...
% \newcommand*\closure[1]{\overline #1}
\newcommand*\closureunion[1]{{#1}\cup \limitset{#1}}
\newcommand*\interior[1]{{#1}^\circ}

% Sets of points
% The set of points within some distance #1 from #2
\newcommand*\neighbor[2]{N_{#1}({#2})}
% The neighborhood without #2
\newcommand*\delneighbor[2]{N_{#1}^*({#2})}
\newcommand*\set[1]{\{ #1 \} }
\newcommand*\conjugate[1]{\overline{#1}}
\newcommand*\sequence[2]{\set{#1}_{#2=1}^\infty}
\newcommand*\series[2]{\sum_{#2=1}^\infty #1_{#2}}
\newcommand*\compose[2]{#1 \circ #2}
\newcommand*\udisk{\mathbb{D}}
\newcommand*\disk[2]{D_{#1}(#2)}
\newcommand*\punctdisk[2]{\disk_{ #1 } - \set{#2}}
\newcommand*\complex{\mathbb{C}}
\newcommand*\naturals{\mathbb{N}}
\newcommand{\integers}{\mathbb{Z}}
\newcommand*\rationals{\mathbb{Q}}
\newcommand*\reals{\mathbb{R}}

% Function spaces
\newcommand*\cnfunc[2]{C^{#2}\left(#1 \right)}
\newcommand*\cnfuncdom[1]{C^{#1}\left(\domain \right)}
\newcommand*\linffunc[1]{L^{\infty}\left(#1 \right)}
\newcommand*\linffuncdom{L^{\infty}\left(\domain \right)}
\newcommand*\lnfunc[2]{L^{#2}\left(#1 \right)}
\newcommand*\lnfuncdom[1]{L^{#1}\left(\domain \right)}
\newcommand*\sobolev[3]{W^{#2, #3}\left(#1 \right)}
\newcommand*\sobolevdom[2]{W^{#1, #2}\left(\domain \right)}
\newcommand*\sobolevh[2]{H^{#2}\left(#1 \right)}
\newcommand*\sobolevhdom[1]{H^{#1}\left(\domain \right)}
\newcommand*\sobolevcs[3]{W_0^{#2, #3}\left(#1 \right)}
\newcommand*\sobolevcsdom[2]{W_0^{#1, #2}\left(\domain \right)}
\newcommand*\sobolevhcs[2]{H_0^{#2}\left(#1 \right)}
\newcommand*\sobolevhcsdom[1]{H_0^{#1}\left(\domain \right)}

\newcommand*\grad{D}
\newcommand*\graddir[1]{D_{#1}}
\newcommand*\lapl{∆}
\newcommand*\diffquot[1]{D^{#1}}
\newcommand*\diffquotdir[2]{D_{#2}^{#1}}

\newcommand*\domain{U}
\newcommand*\bndry[1]{\partial #1}
\newcommand*\bndrydom{\partial \domain}
\newcommand*\compactcont{\subset \subset} % U \compactcont V \rightarrow U \subset \closure{U} \subset V, where U, V are (open) domains

\newcommand*\ballunit{B_1}
\newcommand*\ball[2]{B_{#2}(#1)}
\newcommand*\bunitsurfarea[1]{\omega_{#1}}
\newcommand*\bunitsurfareadef{\omega_n}
\newcommand*\bunitvolume[1]{\alpha_{#1}}
\newcommand*\bunitvolumedef{\alpha_n}

\newcommand*\limitto[2]{\lim \limits_{#1 \rightarrow #2}}

\newcommand{\dd}[1]{\;\mathrm{d}#1}
\newcommand{\dx}{\dd{x}}
\newcommand{\dy}{\dd{y}}
\newcommand{\dz}{\dd{z}}
\newcommand{\dr}{\dd{r}}
\newcommand{\ds}{\dd{s}}
\newcommand{\dS}{\dd{S}}
\newcommand{\dt}{\dd{t}}
\newcommand*\pderiv[2]{\frac{\partial #1}{\partial #2}}
\newcommand*\nthpderiv[3]{\frac{\partial^{#3} #1}{\partial #2^{#3}}}
\newcommand*\deriv[2]{\frac{\dd{#1}}{\dd{#2}}}
\newcommand*\nthderiv[3]{\frac{\dd{^{#3} #1}}{\dd{#2^{#3}}}}

% Norms
\newcommand*\linfnorm[2]{\left\lVert#1\right\rVert_{L^{\infty}(#2)}}
\newcommand*\linfnormdom[1]{\left\lVert#1\right\rVert_{L^{\infty}(\domain)}}

\newcommand*\lnorm[3]{\left\lVert#1\right\rVert_{L^{#2}(#3)}}
\newcommand*\lnormdom[2]{\left\lVert#1\right\rVert_{L^{#2}(\domain)}}

\newcommand*\hnorm[3]{\left\lVert#1\right\rVert_{H^{#2}(#3)}}
\newcommand*\hnormdom[2]{\left\lVert#1\right\rVert_{H^{#2}(\domain)}}

\newcommand*\wnorm[4]{\left\lVert#1\right\rVert_{W^{#2, #3}(#4)}}
\newcommand*\wnormdom[3]{\left\lVert#1\right\rVert_{W^{#2, #3}(\domain)}}

\DeclareMathOperator{\res}{res}
\DeclareMathOperator{\sign}{sign}
\DeclareMathOperator{\diam}{diam}
\DeclareMathOperator{\partition}{Partition}

% Matrix computations
\newcommand*\trace[1]{\text{tr}\left( #1 \right)}

% Average integral from https://tex.stackexchange.com/questions/759/average-integral-symbol
\def\Xint#1{\mathchoice
{\XXint\displaystyle\textstyle{#1}}%
{\XXint\textstyle\scriptstyle{#1}}%
{\XXint\scriptstyle\scriptscriptstyle{#1}}%
{\XXint\scriptscriptstyle\scriptscriptstyle{#1}}%
\!\int}
\def\XXint#1#2#3{{\setbox0=\hbox{$#1{#2#3}{\int}$ }
\vcenter{\hbox{$#2#3$ }}\kern-.6\wd0}}
\def\ddashint{\Xint=}
\def\dashint{\Xint-}
\def\avgint{\dashint}


\begin{document}
\section*{Critical Points and Optimization}
Critical points occur when either all of the partial derivatives go to zero or at least one doesn't exist.
e.g. $f(x, y, z) = \sqrt{|x|} \cos(x) y^2 + y $ has critical points when $x = 0$ and when $|x|^2 \tan(x) = \frac{x}{2}$, $y = 0$

Classification of critical points is done by applying the second derivative test.\\
Define $D = f_{xx}(x_c, y_c) f_{yy}(x_c, y_c) - f_{xy}(x_c, y_c)^2$\\
Then if
\begin{enumerate}
  \item $D > 0$ and $f_{xx}(x_c, y_c) > 0$, then $(x_c, y_c)$ is a local minimum
  \item $D > 0$ and $f_{xx}(x_c, y_c) < 0$, then $(x_c, y_c)$ is a local maximum
  \item $D < 0$, then $(x_c, y_c)$ is a saddle point
  \item $D = 0$, then the second derivative test can't classify the point
\end{enumerate}
\section*{Constrained Optimization and Lagrange Multipliers}
If we're optimizing on a bounded region, e.g. $x^2 + y^2 + z^2 \leq 1$,
we also need to consider what happens on the boundary in addition to the location of critical points.
Recall that the gradient gives the direction of maximum increase of the function we're optimizing,
and that the function is flat in directions perpendicular to the gradient.
Then if we can find points on the boundary where the gradient is perpendicular to the boundary,
we'll have a critical point in the lower dimensional region of the boundary.

More directly, given a boundary $g(x, y, z) = c$ and a function $f(x, y, z)$ to optimize,
we need to find points $(x_c, y_c, z_c)$ s.t. $\nabla f(x_c, y_c, z_c) = \lambda \nabla g(x_c, y_c, z_c)$ for some real $\lambda$.
Note that this is only for one constraint. When multiple constraints $g$ and $h$ hold, we need to find points s.t.
$\nabla f(x_c, y_c, z_c) = \lambda \nabla g(x_c, y_c, z_c) + \mu h(x_c, y_c, z_c)$ for some real $\lambda, \mu$.
This can be visualized as the intersection of two surfaces; generally there's an angle between the surfaces at any point,
if the gradient points in that angle and not along the curve of intersection, we have a potential max or min.
\section*{Problems}
\begin{enumerate}
\item Compute and classify the critical points of $f(x, y) = x^4 - 2 x^2 y^2 + y^4$.
  Then compute the global max and min in the region where $g(x, y) = x^2 + y^2 \leq 1$.
  \newpage
\item Compute and classify the critical points of $f(x, y) = e^{x y}$.
  Then compute the global max and min in $x, y > 0$, $x^3 + y^3 - 16 \leq 0$
  \newline
  \newline
  \newline
  \newline
  \newline
  \newline
  \newline
  \newline
  \newline
  \newline
  \newline
  \newline
  \newline
\item Find the points on the surface $g(x, y, z) = x y^2 z^3 = 8$ which are closest to the origin.
  \newline
  \newline
  \newline
  \newline
  \newline
  \newline
  \newline
  \newline
  \newline
  \newline
  \newline
  \newline
  \newline
\item Find the maximum and minimum heights ($z$) of the curve of intersection of the plane $2 x + y - z = 3$ and the cylinder $2 x^2 + y^2 = 4$.
\end{enumerate}
\end{document}
