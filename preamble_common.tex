
\newcommand*\limitset[1]{{#1}^\prime}
% Not working in unicode math with xelatex for some reason...
% \newcommand*\closure[1]{\overline #1}
\newcommand*\closureunion[1]{{#1}\cup \limitset{#1}}
\newcommand*\interior[1]{{#1}^\circ}

% Sets of points
% The set of points within some distance #1 from #2
\newcommand*\neighbor[2]{N_{#1}({#2})}
% The neighborhood without #2
\newcommand*\delneighbor[2]{N_{#1}^*({#2})}
\newcommand*\set[1]{\{ #1 \} }
\newcommand*\conjugate[1]{\overline{#1}}
\newcommand*\sequence[2]{\set{#1}_{#2=1}^\infty}
\newcommand*\series[2]{\sum_{#2=1}^\infty #1_{#2}}
\newcommand*\compose[2]{#1 \circ #2}
\newcommand*\udisk{\mathbb{D}}
\newcommand*\disk[2]{D_{#1}(#2)}
\newcommand*\punctdisk[2]{\disk_{ #1 } - \set{#2}}
\newcommand*\complex{\mathbb{C}}
\newcommand*\naturals{\mathbb{N}}
\newcommand{\integers}{\mathbb{Z}}
\newcommand*\rationals{\mathbb{Q}}
\newcommand*\reals{\mathbb{R}}

% Function spaces
\newcommand*\cnfunc[2]{C^{#2}\left(#1 \right)}
\newcommand*\cnfuncdom[1]{C^{#1}\left(\domain \right)}
\newcommand*\linffunc[1]{L^{\infty}\left(#1 \right)}
\newcommand*\linffuncdom{L^{\infty}\left(\domain \right)}
\newcommand*\lnfunc[2]{L^{#2}\left(#1 \right)}
\newcommand*\lnfuncdom[1]{L^{#1}\left(\domain \right)}
\newcommand*\sobolev[3]{W^{#2, #3}\left(#1 \right)}
\newcommand*\sobolevdom[2]{W^{#1, #2}\left(\domain \right)}
\newcommand*\sobolevh[2]{H^{#2}\left(#1 \right)}
\newcommand*\sobolevhdom[1]{H^{#1}\left(\domain \right)}
\newcommand*\sobolevcs[3]{W_0^{#2, #3}\left(#1 \right)}
\newcommand*\sobolevcsdom[2]{W_0^{#1, #2}\left(\domain \right)}
\newcommand*\sobolevhcs[2]{H_0^{#2}\left(#1 \right)}
\newcommand*\sobolevhcsdom[1]{H_0^{#1}\left(\domain \right)}

\newcommand*\grad{D}
\newcommand*\graddir[1]{D_{#1}}
\newcommand*\lapl{∆}
\newcommand*\diffquot[1]{D^{#1}}
\newcommand*\diffquotdir[2]{D_{#2}^{#1}}

\newcommand*\domain{U}
\newcommand*\bndry[1]{\partial #1}
\newcommand*\bndrydom{\partial \domain}
\newcommand*\compactcont{\subset \subset} % U \compactcont V \rightarrow U \subset \closure{U} \subset V, where U, V are (open) domains

\newcommand*\ballunit{B_1}
\newcommand*\ball[2]{B_{#2}(#1)}
\newcommand*\bunitsurfarea[1]{\omega_{#1}}
\newcommand*\bunitsurfareadef{\omega_n}
\newcommand*\bunitvolume[1]{\alpha_{#1}}
\newcommand*\bunitvolumedef{\alpha_n}

\newcommand*\limitto[2]{\lim \limits_{#1 \rightarrow #2}}

\newcommand{\dd}[1]{\;\mathrm{d}#1}
\newcommand{\dx}{\dd{x}}
\newcommand{\dy}{\dd{y}}
\newcommand{\dz}{\dd{z}}
\newcommand{\dr}{\dd{r}}
\newcommand{\ds}{\dd{s}}
\newcommand{\dS}{\dd{S}}
\newcommand{\dt}{\dd{t}}
\newcommand*\pderiv[2]{\frac{\partial #1}{\partial #2}}
\newcommand*\nthpderiv[3]{\frac{\partial^{#3} #1}{\partial #2^{#3}}}
\newcommand*\deriv[2]{\frac{\dd{#1}}{\dd{#2}}}
\newcommand*\nthderiv[3]{\frac{\dd{^{#3} #1}}{\dd{#2^{#3}}}}

% Norms
\newcommand*\linfnorm[2]{\left\lVert#1\right\rVert_{L^{\infty}(#2)}}
\newcommand*\linfnormdom[1]{\left\lVert#1\right\rVert_{L^{\infty}(\domain)}}

\newcommand*\lnorm[3]{\left\lVert#1\right\rVert_{L^{#2}(#3)}}
\newcommand*\lnormdom[2]{\left\lVert#1\right\rVert_{L^{#2}(\domain)}}

\newcommand*\hnorm[3]{\left\lVert#1\right\rVert_{H^{#2}(#3)}}
\newcommand*\hnormdom[2]{\left\lVert#1\right\rVert_{H^{#2}(\domain)}}

\newcommand*\wnorm[4]{\left\lVert#1\right\rVert_{W^{#2, #3}(#4)}}
\newcommand*\wnormdom[3]{\left\lVert#1\right\rVert_{W^{#2, #3}(\domain)}}

\DeclareMathOperator{\res}{res}
\DeclareMathOperator{\sign}{sign}
\DeclareMathOperator{\diam}{diam}
\DeclareMathOperator{\partition}{Partition}

% Matrix computations
\newcommand*\trace[1]{\text{tr}\left( #1 \right)}

% Average integral from https://tex.stackexchange.com/questions/759/average-integral-symbol
\def\Xint#1{\mathchoice
{\XXint\displaystyle\textstyle{#1}}%
{\XXint\textstyle\scriptstyle{#1}}%
{\XXint\scriptstyle\scriptscriptstyle{#1}}%
{\XXint\scriptscriptstyle\scriptscriptstyle{#1}}%
\!\int}
\def\XXint#1#2#3{{\setbox0=\hbox{$#1{#2#3}{\int}$ }
\vcenter{\hbox{$#2#3$ }}\kern-.6\wd0}}
\def\ddashint{\Xint=}
\def\dashint{\Xint-}
\def\avgint{\dashint}
